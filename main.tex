\documentclass{sig-alternate}


\usepackage{enumitem}
\usepackage{framed}
%\usepackage[11pt]{moresize}
\usepackage{cprotect}
\usepackage{enumitem}
\usepackage{listings}
\usepackage{amstext}
\usepackage{amstext}
\usepackage{pdfpages}
\usepackage{alltt}
\usepackage{epstopdf}
\usepackage{xspace,colortbl}
\usepackage[USenglish]{babel}
\usepackage{multirow}
\usepackage[hyphens]{url}
\usepackage{subfigure}
\usepackage{graphicx}%%
\usepackage{amssymb}
\usepackage{fmtcount}
\usepackage{amsfonts}
\usepackage{xspace}
\usepackage{amsmath}
\usepackage{multirow}
\usepackage[mathscr]{eucal}
%\usepackage{psfrag}
\usepackage{colortbl}

\usepackage{amsmath,amssymb}
\usepackage[linesnumbered, ruled,vlined]{algorithm2e}

\usepackage{caption}
\usepackage{graphicx}

\usepackage{bm}
\usepackage{charter}
\usepackage[nospace]{cite}
\usepackage{csquotes}
\usepackage{enumitem}

\lstset{basicstyle=\small,breaklines=true,language=Python, frame=single}

\usepackage{balance}

%\linespread{0.99}

\makeatletter
\def\@copyrightspace{\relax}
\makeatother


\DeclareMathOperator*{\argmin}{arg\,min}
\DeclareMathOperator*{\argmax}{arg\,max}


\begin{document}

%\setlength{\belowdisplayskip}{3pt} \setlength{\belowdisplayshortskip}{3pt}
%\setlength{\abovedisplayskip}{3pt} \setlength{\abovedisplayshortskip}{3pt}
%\setlength{\belowcaptionskip}{-10pt}
%\selectfont

\newtheorem{theorem}{Theorem}
\newtheorem{example}{Example}
\newtheorem{definition}{Definition}
\newtheorem{problem}{Problem}
\newtheorem{property}{Property}
\newtheorem{proposition}{Proposition}
\newtheorem{lemma}{Lemma}
\newtheorem{corollary}{Corollary}

\newcommand{\company}{\texttt{Company X}}
\newcommand{\cond}{\textrm{pred}\xspace}
\newcommand{\dataset}{data set\xspace}
\newcommand{\datasets}{data sets\xspace}
\newcommand{\spview}{\textsf{SPView}\xspace}
\newcommand{\fjview}{\textsf{FJView}\xspace}
\newcommand{\aggview}{\textsf{AggView}\xspace}
\newcommand{\hashfunc}[1]{\textsf{hash}(#1)\xspace}
\newcommand{\hashop}{\textsf{hash}\xspace}
\newcommand{\nsc}{\textsf{NormalizedSC}\xspace}
\newcommand{\rsc}{\textsf{RawSC}\xspace}

\newcommand{\avgfunc}{\ensuremath{\texttt{avg} }\xspace}
\newcommand{\maxfunc}{\ensuremath{\texttt{max} }\xspace}
\newcommand{\minfunc}{\ensuremath{\texttt{min} }\xspace}
\newcommand{\histfunc}{\ensuremath{\texttt{histogram\_numeric} }\xspace}
\newcommand{\countfunc}{\ensuremath{\texttt{count}}\xspace}
\newcommand{\sumfunc}{\ensuremath{\texttt{sum} }\xspace}
\newcommand{\varfunc}{\ensuremath{\texttt{var} }\xspace}
\newcommand{\stdfunc}{\ensuremath{\texttt{std} }\xspace}
\newcommand{\covfunc}{\ensuremath{\texttt{cov} }\xspace}
\newcommand{\corrfunc}{\ensuremath{\texttt{corr} }\xspace}
\newcommand{\medfunc}{\ensuremath{\texttt{median} }\xspace}
\newcommand{\percfunc}{\ensuremath{\texttt{percentile} }\xspace}
\newcommand{\havingfunc}{\ensuremath{\texttt{HAVING} }\xspace}
\newcommand{\selectfunc}{\ensuremath{\texttt{select} }\xspace}
\newcommand{\ratio}{\ensuremath{\rho }\xspace}


\newcommand{\insertion}{\ensuremath{\texttt{INSERT} }\xspace}
\newcommand{\update}{\ensuremath{\texttt{UPDATE} }\xspace}
\newcommand{\delete}{\ensuremath{\texttt{DELETE} }\xspace}

\newcommand{\sysfull}{BoostClean\xspace}
\newcommand{\sys}{BoostClean\xspace}
\newcommand{\sysnospace}{BoostClean}

\newcommand{\tbl}[1]{\textsf{#1}\xspace}
\newcommand{\field}[1]{\textsf{#1}\xspace}
\newcommand{\cost}{\textrm{cost}\xspace}
\newcommand{\ans}{\textsf{ans}\xspace}
\newcommand{\dans}{\Delta\textsf{ans}\xspace}
\newcommand{\cqp}{correction query processing\xspace}
\newcommand{\Cqp}{Correction query processing\xspace}

\newcommand{\reminder}[1]{{{\textcolor{magenta}{\{\{\bf #1\}\}}}\xspace}}
\newcommand{\ewu}[1]{{{\textcolor{blue}{\{\{\bf ewu:\} #1\}}}\xspace}}
\newcommand{\mps}[1]{{{\textcolor{red}{\{\{\bf meelap:\} #1\}}}\xspace}}

\definecolor{light-gray}{gray}{0.95}
\definecolor{mid-gray}{gray}{0.85}
\definecolor{green}{RGB}{0,176,80}
\definecolor{darkred}{rgb}{0.7,0.25,0.25}
\definecolor{darkgreen}{rgb}{0.15,0.55,0.15}
\definecolor{darkblue}{rgb}{0.1,0.1,0.5}
\definecolor{orange}{RGB}{237,125,49}
\definecolor{blue}{RGB}{68,114,196}
\newcommand{\blue}[1]{{\textcolor{blue}{{\bf #1}}\xspace}}
\newcommand{\orange}[1]{{\textcolor{orange}{{\bf #1}}\xspace}}



\newcommand{\specialcell}[2][c]{%
  \begin{tabular}[#1]{@{}c@{}}#2\end{tabular}}

\def\ojoin{\setbox0=\hbox{$\bowtie$}%
  \rule[-.02ex]{.25em}{.4pt}\llap{\rule[\ht0]{.25em}{.4pt}}}
\def\leftouterjoin{\mathbin{\ojoin\mkern-5.8mu\bowtie}}
\def\rightouterjoin{\mathbin{\bowtie\mkern-5.8mu\ojoin}}
\def\fullouterjoin{\mathbin{\ojoin\mkern-5.8mu\bowtie\mkern-5.8mu\ojoin}}

%\setlength{\belowcaptionskip}{-10pt}

%\newcommand{\reminder}[1] {}
\pagestyle{plain}

%\input{coverletter.tex}

%\title{ActiveClean: Progressive Data Cleaning For Convex Data Analytics}
\title{\sys: Learning an Ensemble of Data Cleaning Models to Optimize Machine Learning}


%\fontsize{9pt}{11pt}
%\selectfont


\maketitle

\begin{abstract}
As Machine Learning scales up to larger datasets, the challenges in handling dirty data during training and prediction become more significant.
Currently, it requires painstaking human effort to monitor and maintain learning pipelines to ensure that prediction accuracy does not drop due to unexpected data.
We present a system called \sys that automatically detects data errors, and selects repairs that maximally improve predictive performance of a user-defined ML model.
Since repairs are potentially ambiguous, \sys ``hedge its bets'' by using an ensemble of models trained with different repair strategies.
We describe materialization, indexing, and parallelization strategies that find this ensemble efficiently, and results suggest greater than an order-of-magnitude speedup.
\sys provides a library of dataset-independent error detectors
based on deterministic heuristics and statistical criteria. To better detect errors in categorical attributes, one of the modules in this library is a novel anomaly detector based on the Word2Vec Neural Network architecture.
We evaluated results on 8 ML datasets on Kaggle and the UCI repository with real data errors and compare to statistical anomaly detection techniques, constraint-based techniques, and the best single cleaner performance.
We demonstrate how we can parallelize the inner-loop of the boosting operation, and on a 16-core machine \sys achieves a 9.7x speedup.
\end{abstract}


%\pagenumbering{gobble}


\section{Introduction}\label{intro}\sloppy
The availability of data and vast cloud-based computational resources has ushered in an era of more sophisticated machine learning (ML) models in prediction, recommendation, and automation.
The database community has built systems to support almost every stage of the development process including featurization~\cite{keystone,zhang2014mat}, distributed model training~\cite{hellerstein2012madlib, crotty2014tupleware, feng2012towards, tensor}, and model deployment~\cite{crankshawmissing}.  
However, an under-served, yet crucial, component is the management and cleaning of dirty data. 
 If unaccounted for, this dirty data can drastically bias predictions that are undesirable or even dangerous~\cite{vanderbilt2012let}.  Recent papers and surveys of analysts suggest that such problems are pervasive~\cite{sculley2014machine,kandel2012,krishnan2016hilda}.

As a concrete example, we are collaborating with a data science company called \company\footnote{Anonymized at the request of the company.} that ranks sales leads based on Salesforce.com data on past sales leads, and additional information scraped from the web about the client.
The company predicts the probability of viability for a potential (unlabeled) lead.
The data are acquired from a combination of manual data entry and automatically scraped web sources, and thus, inconsistencies, missing data, and incorrect values are a significant problem.  For instance, a typical error is the inconsistent representation of missing values (e.g., ``-999'', ``EMPTY'' or ``none'' may be used depending on the sales representative).  If the featurization code does not recognize and address these errors, it can lead to biases that degrade the quality of the model. For example, the data scientist may impute a default mean value for all blank attributes but miss the code ``-999'', which is then interpreted as a semantic value. 
Detecting and repairing all such errors is extremely time-consuming, and for every new client this effort will have to be repeated.

This company's data cleaning challenges are not unique and are prevalent in many industrial ML pipelines~\cite{krishnan2016hilda}.  
Software Engineers write custom conditional cleaning scripts that are a combination of a {\it detector}, which are a collection of Boolean functions that specify a subset of records that are dirty, and {\it repair} functions that transform or delete those records.  It is not enough to write these scripts once.
The predictive nature of ML applications means that the system will continuously encounter and process new, unseen data.
Software Engineers must constantly monitor and maintain the data processing pipeline to account for unexpected changes~\cite{sculley2014machine, DBLP:conf/sigmod/KrishnanFGWW16}.
To further exacerbate this problem, modern prediction models rely on data integrated from a wide variety of sources (e.g., \company combines on average 5-10 sources to train a model).  For each data source, the engineer must understand domain-specific information (e.g., invariants) in order to accurately clean the data.  For instance, we found that each machine learning dataset required between $1-7$ custom error detection rules in order to identify the low-hanging errors in those datasets.

To reduce this burden, we present a new system, called \sys, that automates the process of detecting and repairing a common class of data errors called {\it domain value violations} that occur when an attribute value is outside of its value domain.  Numerous data quality surveys across the database, statistics, and scientific literature highlight the prevalence, variety, and importance of this class of errors, which include missing data, incorrect data, or inconsistent representations of the same logical data value~\cite{muller2005problems,li2010improving,kim2003taxonomy,kandel2011research}.  \sys focuses on this common class of errors, and leaves more complex scenarios such as entity resolution to the Software Engineers. After deployment, \sys can help ensure that deployed models maintain high accuracy even in the presence of incoming dirty data, and engineers are only needed to address drastic changes to the input data. 

In traditional relational data cleaning, it is very hard to quantify the accuracy of an automatic data cleaning process without ground truth--a dataset where {\it all attributes are fully correct}.
On the other hand, in ML, cleanly labeled test data is often available (e.g., the results of following a sales lead). 
Labels often represent directly observed phenomena making them relatively clean, while features are often weaker signals integrated from multiple disparate sources and subject to error and frequent change.
This allows us to define accuracy in terms of the model's predictive accuracy--the data cleaning being a means to improving that predictive accuracy.
In this sense, our goal is not to fully clean each record and recover a consistent relation; instead, to utilize the available cleaning resources to best improve a model trained on this dataset.
The key challenge is to efficiently search the space of possible conditional data cleaning scripts (detector and repair combinations) while ensuring that the model does not overfit~\cite{DBLP:journals/pvldb/KrishnanWWFG16,krishnan2016hilda}.   

Our primary observation is that a conditional cleaning script can be interpreted as generating a new set of features (the cleaned values), and thereby generating a new model trained on those cleaned features. 
We can view the process of selecting the best sequence of cleaning operations as an ensembling problem, i.e., selecting the best collection models that collectively estimate a label. 
Although there are many possible algorithms~\cite{dietterich2000ensemble}, we use a powerful technique called Boosting~\cite{freund1995desicion} that composes a set of weak learners into a strong learner.  
First, unlike methods that are specific to certain classes of models (e.g., linear models, differentiable models), boosting can be applied to black-box models. 
Second, it takes interactions and correlations between the different data cleaning models into account by incrementally selecting ``orthogonal'' compositions.


\sys takes as input a relational table, a library of detector functions $\mathcal{D}$ that generate (possibly incorrect) predicates that match candidate dirty records, a library of repair functions $\mathcal{F}$ that transform or delete a record, and a user-specified classifier training procedure \texttt{train()}.
\sys has two key components: an automatic error detector to determine subsets of records that are dirty, and a repair selector to select repair actions for those dirty records using boosting.
We cast the former component into a featurization problem so that the user focuses on the familiar task of creating feature extraction functions while \sys translates these features into error detection rules using a technique called Isolation Forests~\cite{liu2008isolation}.  Further, we have written a general set of featurizers, including one that is a novel adaptation of the \textsf{word2vec} neural network architecture that is effective at detecting multi-attribute errors.  The neural network can be individually tailored to each dataset and learn to predict the co-occurrence of attributes in a record. 
The detectors output relational predicates $p_i$, which can be used to detect candidate errors.  The second component then uses boosting to generate a sequence of conditional cleaning scripts $(p_i, r_i)$ to be applied to the training and test datasets, where $r_i$ is the repair function to be applied to records matching predicate $p_i$.

This paper focuses on data errors that cause domain value violations in the context of supervised classification models (both single and multi-class).  The system is currently designed for a single-node setting. Our contributions are as follows:

\vspace{0.25em}\noindent\textbf{Cleaning as Boosting: } We present a new automated data cleaning system based on statistical boosting that finds the best ensemble of operations from a library of operations to maximize the predictive performance of a downstream model. 

\stitle{Automatic Model Improvements:} We evaluated \sys on 12 datasets collected from Kaggle, the UCI repository, real-world data analyses, and \company, and improved absolute prediction accuracy by up to $9\%$ in comparison to baseline (non-ensembled integrity constraint+quantitative outlier detector) approaches on completely unseen test data. 

\vspace{0.5em}\noindent\textbf{Error Detection Library: } We have built an optimized library of data cleaning operations based on deterministic rules and statistical criteria from which \sys selects. To better detect errors in categorical attributes, we developed a novel detector based on the \textsf{Word2Vec} neural network architecture. Following prior experimental procedures~\cite{DBLP:journals/pvldb/AbedjanCDFIOPST16}, the library achieves a detection accuracy of 81\% of all of the errors found by hand-written rules on eight machine learning datasets.  %\sys is on average 40\% more accurate than applying statistical outlier detection to only the quantitative attributes.

\vspace{0.5em}\noindent\textbf{Optimizations: } Our optimizations including parallelism, materialization, and indexing techniques show a $22.2\times$ end-to-end speedup on a 16-core machine.

% and indexing-based optimizations speed up the boosting and repair selection  We demonstrate how we can parallelize the inner-loop of the boosting operation, and on a 16-core machine \sys achieves a 9.7x speedup for the repair selection step. Similarly, we show that building an index can speed up operator selection .






\iffalse
The problem of dirty training data in ML is subtle as most learning algorithms are robust to statistical noise.
However, un-modeled systematic biases in the training data can still adversely affect the results~\cite{DBLP:journals/pvldb/KrishnanWWFG16, DBLP:conf/case/MahlerKLSMKPWFAG14, xiaofeature}.
The way that the developer chooses to address corruption will have a significant impact on the performance of the ML application.
Consider a music recommender system where a recent software update causes songs longer than 5 minutes to have ``NULL'' ratings.
If the ML developer treats a NULL rating as ``0 stars'', those songs may never get recommended.
To avoid this bias, it may be more prudent to discard those ratings or impute a default  value (e.g., mean over all previous non-NULL ratings).

To setup the abstract search problem, we are given a dataset $R$, a library of data cleaning operations $\mathcal{L}=\{l_1,...,l_k\}$, a user-specified model training program which returns a classifier, and an oracle that evaluates the prediction accuracy of the classifier (e.g., a ground truth clean test dataset).
Our objective is to find a classifier that maximizes prediction accuracy by applying compositions of operators in our library to $R$ and training on the resulting dataset.
While this problem is inherently combinatorial, the key insight is to model the hypothesis testing procedure as a form of adaptive statistical boosting. 
\fi




%The datasets are inconsistent in the way they represent missing information (e.g., some numerical fields left blank, some fields with a placeholder value of ``-999''). 
%Featurization code that does not recognize that a blank attribute value is semantically equivalent to a ``-999'' attribute value can lead to biases--for example, the data scientist may impute a sensible default mean value for all of the blank attributes but treat the ``-999'' as the given value.

\if{0}
Clearly, some level of automation in detecting and handling erroneous data can reduce the burden on data scientists.
Automated rule-based data repair is a well-studied field~\cite{DBLP:conf/sigmod/ChuIKW16}, but the ML setting presents additional challenges and structure that are important to understand.
In ML, incoming records are, in a sense, both data (during training) and queries (during prediction).
This provides additional degrees-of-freedom in handling dirty data.
For example, when asked to predict a label for a dirty example, one may want to return a ``fail-safe'' prediction instead of cleaning it first and then asking for a prediction.
Second, ML applications often have a way of measuring prediction accuracy.
Labels often represent directly observed user-behavior (e.g., sale vs. no sale, whether the user clicked a link etc.), and thus, are relatively consistent over the lifetime of an application.
On the other hand, the features used to predict the labels may be integrated from a variety of different company databases and susceptible to inconsistencies and change.
With this in mind, we present \sys, a new data cleaning system that detects errors in ML data and uses knowledge of the labels to adaptively select from a set of repair actions to maximize prediction accuracy.
\fi
\vspace{2em}
\section{Background}
This section motivates \sys in relation to prior work.
Using the pilot study as inspiration, consider the following running example to describe the system and notation:

\begin{example}[Lead Prediction]\sloppy\label{ex:lead}
Past clients are stored in a relational database:
\[
R(id, name, num\_emp, industry, region, successful)
\]
where $name$ is the company name, $num\_emp$ is the number of employees in the company, $industry$ is a categorical attribute that describes the industry segment, $region$ is a code indicating the region of the country the business is headquartered, and $is\_successful$ is a Boolean describing whether the company purchased the product.
\end{example}

\begin{figure}[t]
% \vspace{-5pt}
\centering
 \includegraphics[width=\columnwidth]{figures/training_and_pred_errors.png}
 \caption{The above example uses the invariant that the number of employees is greater than $0$.  Training errors are violations of the invariante in the training dataset (top row). Prediction errors are invariant violations in the test data during prediction. \sys is a tool to {\it detect} both types of errors and generate a corrective {\it repair} action.
 \label{fig:error}}
\end{figure}

\subsection{Machine Learning and Dirty Data}
It is important to emphasize that we are not concerned about statistical noise or the natural variance of data in the dataset.
Rather, we are interested in cases where an invariant that is expected to be true, i.e., a logical assertion about properties of the dataset, is violated leading unforeseen biases in the way the records are featurized.
For example, suppose we expect every $num\_emp \ne~NULL$, but the particular dataset we receive has null values.
Either the featurization program will error, or worse, it may not error and interpret $NULL$ as $0$.
However, semantically $NULL$ and $0$ refer to two very different concepts: a missing value or a company with no employees.
Conversely, one can also construct cases where two values that have the same semantic meaning are mapped to different numerical representations.
Suppose the attribute $region$ has two representations for the western United States: $USWEST$ and $USW$. 
This error would lead to two different feature values for records from the same region.

Fundamentally, performance of a model degrades when these violated invariants create a mismatch between the training and prediction environments~(Figure \ref{fig:error}):

\vspace{0.5em}\noindent\textbf{Training Errors: } Invariants that are expected to be true during prediction time are violated in the training dataset. In the running example, consider the case when $R$ is integrated from multiple previously collected datasets, where some of the datasets failed to record $num\_emp$. Assuming that there is no convenient way to retroactively determine the ground truth, the analyst has a couple of options to handle this error: (1) drop all records with a missing  $num\_emp$ value, or (2) impute $num\_emp$ with a sensible default value.

\vspace{0.25em}\noindent\textbf{Prediction Errors: } Invariants that are expected to be true during training time are violated when the model is asked to predict a label for a new example. 
Consider the case where all of the training examples recorded $num\_emp$, but a new test example has a null value.
Unlike in the training set, it may not be an acceptable solution to drop a prediction (i.e., the system fails to respond to a query).
So, the analyst has a couple different options to handle this error: (1) impute $num\_emp$ with a sensible default value and then predict, (2) apply a ``fail-safe'' prediction.

\ewu{why say ``expected to be true during training are violated in prediction'' rather than ``invariants are violated''?}

\vspace{0.5em}

Eventually, we may want to integrate the new examples into the model and re-train--turning prediction errors into future training errors.
Already, for a missing value in $num\_emp$, there are four different choices the analyst can handle a violation (2 training $\times$ 2 prediction).
These four choices have to be evaluated for each possible error type based on its severity, data type, and predictive importance.
Recent surveys of data scientists suggest that the current procedure for making these choices is manual and ad-hoc~\cite{kandel2012, krishnan2016hilda}.
As ML applications are increasingly in the critical path, management systems that are resilient to errors by automatically making such choices are important.
\ewu{Not clear what the above text is trying to argue -- that there are too many options to manually try?  that it's not clear what a ``goodness'' criteria is?  somethingelse?}

\subsection{Possible Approaches}
\sys brings together generic, dataset-independent error detection with automatically learned repair strategies for ML applications.
We review how baseline techniques proposed in prior work that could apply to this problem.

\vspace{0.5em}

\noindent\textbf{Rule-based Repair: } 
The traditional relational approach to this problem to ignore the downstream model and clean the relation completely.
In the running example, suppose some of the values for $num\_emp$ are NULL and we want to train a classifier to predict $is\_successful$.
We would define a domain integrity constraint $num\_emp \ne~NULL$, and then propose a set of repairs to satisfy this constraint.
With no other information, this rule-based approach could in principle impute any non-null value from the domain to create a logically consistent relation.
To avoid this problem, we can adopt an approach like~\cite{prokoshyna2015combining} and select the imputations that minimize the statistical distance of the updated relation to an ideal distribution for the attribute, for example, an ideal power-law distribution.
This would impute values in such a way that $num\_emp$ matched a Zipfian distribution.

On its own, this optimization approach has been empirically very successful, however, when we train a classifier on this data, counter-intuitive effects can occur.  
The data cleaning operation may break important correlations in the data and may introduce biases into the training data not present in test conditions. 
Consider the degenerate case where $num\_emp = NULL$ is perfectly correlated with one of the prediction classes--in this case, it may be better to NOT clean the data!
Similarly, consider the case where there is a very strong class imbalance.
When observing a record with $num\_emp = NULL$, rather than making a noisy prediction, it might be better to default to the most popular class. 
While more sophisticated statistical imputation techniques exist~\cite{schafer1998multiple}, they all have the same fundamental problem that the value imputation is divorced from the downstream classifier's predictive accuracy. 
We see this problem in our experiments (Section~\ref{exp:comp}), where on some datasets imputing the most frequent value leads to a more accurate downstream classifier than imputing to minimize the difference from an ideal distribution.

\vspace{0.5em}\noindent\textbf{Statistical Detection: } Rule-based techniques are dependent on the analyst defining the invariants, which might be challenging if the analyst cannot anticipate how future data might look.
There is a well-established line of literature on statistical anomaly detection~\cite{hellerstein2008quantitative}, and for the most part, these techniques are generic and dataset independent (up-to hyperparameters). Typically, such approaches identify \emph{outlier} records outside of some normal range of variance. However, the problem is that not all dirty data look like outliers. In the running example, there could truly be companies where $num\_emp = 0$. It has been shown that statistical anomaly detection techniques miss obvious errors in heterogeneous datasets with mixes of numerical, categorical, and string-valued data~\cite{DBLP:journals/pvldb/AbedjanCDFIOPST16}.

Abedjan et al. recently evaluated a wide range of error detection techniques on 5 proprietary real-world datasets~\cite{DBLP:journals/pvldb/AbedjanCDFIOPST16}.  They found that the errors in 3 of the datasets were dominated by missing cell values; 1 dataset contained functional dependency violations due to erroneous numerical attribute values, and only 1 dataset require complex user-specified denial constraints~\cite{chu2013discovering} to identify the errors.  These findings suggest that, in 4 of the 5 datasets, a significant portion of data errors can be classified as {\it domain integrity errors}, wherein a cell contains a value outside of its domain of permissible values.

We believe this highlights the potential value of automated cleaning systems such as \sys to identify the bulk of common-case errors, so that developers may focus on the specialized, domain-specific errors.  The prevalence of {\it domain integrity errors} suggests that a pre-defined set of featurizers and detector generators can be sufficient to detect these errors.  In fact, on 8 of our experimental datasets, \sys using our pre-populated detector library achieves a detection accuracy of 81\% of all of the errors found by hand-written rules.

Therefore, we need a mix of statistical rules and logic rules to determine errors.
We explore to what extent we can derive these rules from data for routine errors. 
We surveyed 8 ML datasets used in Kaggle competitions and benchmarks in the UCI ML repository, and found that a majority of the non-statistical errors could be detected as \emph{domain integrity constraints}, i.e., disallowed values in single columns.
We apply a combination of heuristic checks for missing values and data type errors, and a neural network based error detector that identifies attribute values not likely to co-occur in the same record.

\iffalse
\subsection{Solution Overview}
There does not exist one single error detector or repair action that dominates, so which ones should an analyst choose? 
\sys models this problem as a boosting problem.
Rather than thinking of each of detector and repair pair as a data transformation, it thinks of each as generating a new ``classifier'' that provides some additional information about the label.
The problem of selecting the top data pairs is equivalent to ensembling a subset of the classifiers as best as possible.
We choose a boosting framework to represent this problem due to the relatively minimal assumptions about the structure and implementation details of the user-specified classifier.
To construct the underlying library of cleaning operations we surveyed datasets on Kaggle and the UCI repository and built a library of data cleaning operations that supported common data cleaning tasks across the datasets.
\fi

\section{Problem Statement}
We now present the formal problem statement along with our assumptions.

\subsection{Problem Setup}
\sys takes as input a dirty training dataset $(X_{train}, Y_{train})$ where both the features $X_{train}$ and labels $Y_{train}$ may have errors, as well as a test dataset $(X_{test}, Y_{test})$ where the features may contain errors however the labels $Y_{test}$ are correct.  Although the training labels may contain errors, the test labels must be clean to ensure an unbiased measure of accuracy that is not affected by data cleaning operations.  Such labels may be collected as part of a gold standard dataset~\cite{marcus2015crowdsourced} or by cross-referencing the data with other sources~\cite{li2012truth}.  
Labels often represent directly observed phenomena such as (e.g., purchased/not purchased), while features are integrated from multiple disparate sources and subject to frequent change.
Let a record $r_i = (x_i,y_i) \in (X_{train},Y_{train})$ denote the features along with its corresponding (possibly null) label, and $r_i.y$ denote the label for the record.    Furthermore, the features may be categorical, or string-valued, in addition to numerical.

\begin{example}[Notation]
In Example~\ref{ex:lead}, the attributes $name$, $n\_emp$, $industry$, and $region$ define the schema of $X_{train,test}$, and the attribute $successful$ corresponds to the labels $Y_{train,test}$.
\end{example}

Let a classifier $C(r_i) = r_i'$ be a function that takes as input a record $r_i$ and sets $r_i.y$ to the predicted label value.
A classifier predicts $(x_i, y_i) \in (X_{test}, Y_{test})$ correctly if  $C((x_i, null)).y = y_i$.  $C$'s test accuracy is defined as the fraction of correctly predicted test records:
\[
acc(C) = \frac{|\{\forall x,y \in (X_{test}, X_{test})~:~ C((x, null)).y = y\}|}{|Y_{test}|}
\]
To generate a classifier, the user provides \textsf{train}($X_{train}, Y_{train}$) that return a classifier $C$. We model \textsf{train}($\cdot$) as a black-box and assume that the function internally performs any necessary featurization.

\begin{example}[Classification]\sloppy
The classifier $C$ can be a support vector machine predicting whether $successful = true$ based on a feature vector derived from
$name$, $n\_emp$, $industry$, and $region$.
\end{example}

\subsection{Detection and Repair Libraries}\label{s:detectorgen}
We assume that the user provides a library of detector generators $\mathcal{D} = \{d_1,\cdots\}$ and a repair library $\mathcal{F} = \{f_1,\cdots\}$.  \sys uses $\mathcal{D}$ to generate predicates that identify candidate dirty records, and selects the appropriate repair functions in $\mathcal{F}$ to those records.  

\subsubsection{Detection Generators and Predicates}
We define a predicate $p_i$ as a Boolean expression over an input record that returns the set of referenced attributes if it evaluates to $true$ and an empty set otherwise.  Based on this definition, we say that $r$ is a candidate dirty record if $p_i(r) \ne \emptyset$. For instance, $p_i(r) = r.n\_emp \le 0$ is an example of the former: if a company record contains $0$ employees, then the predicate will return $\{n\_emp\}$.  From an API perspective, we need a more expressive model than pre-defined Boolean expressions: 

First, predicate expressions may reference combinations of attributes.  For instance, if we knew that there are no oil and natural gas companies in the northwest, the predicate $p_i(r) =  (r.region == USNW \wedge r.industry \in ('OIL','NG'))$ would return $\{region, industry\}$ if such a company were detected.
Second, predicates may apply transformation functions over the input data.  For instance, the following predicate first featurizes the record using a function $g$, and applies a threshold to the first element of the feature vector: $p_i(r) = g(r)[0] > 10$.  
Third, predicate expressions may contain aggregate expressions that are computed over all records in the training dataset $X_{train}$.  For instance, the following predicate performs Quantitative Error Detection~\cite{hellerstein2008quantitative} by checking whether the record's $n\_emp$ value is further than $5$ standard deviations of the mean: 
$$p_i(r) = |r.n\_emp - avg(r.n\_emp)| > 5\times stddev(r.n\_emp)$$

To address these problems we define a detector generator $d_i$ is a function that takes the full training set as input and returns a predicate $p_i$. 
In thise sense, predicates can be derived or learned from previous data.

\subsubsection{Repair Functions}
Each repair function $f_i \in \mathcal{F}$ is a function that takes a record as input and modifies the record's attributes.  We consider two types of repairs:  {\it data repairs} are applied to the training data prior to running the training procedure, while {\it prediction repairs} modify the label of the records {\it after} the classifier makes a prediction.   

{\it Data repairs} modify the values of a training record in response to a detected error (due to a predicate).  These repair functions are free to modify the record's features, label, or simply delete the record from the training dataset.  

{\it Prediction repairs}, on the other hand, take as input the non-transformed record along with the classifier prediction, and replaces the prediction with a default value.  This is useful when the input record is too corrupted to provide a reliable prediction.  For instance, the NFL play-by-play dataset describe in Section~\ref{s:exp}, some input records contain almost all null attributes and it is more accurate to default the prediction to the most frequent label rather than attempting a repair.

Note that this section formalizes an API for these operations and subsequent sections provide one instantiation of this library.

\subsubsection{Conditional Repairs}
\sys applies repair functions to specific sets of records through the use of {\it conditional repairs}.  A conditional repair $l_k = (p_k, f_k)$ is a tuple where $p_k = d_i(X_{train}, Y_{train})$ is the output of a detector generator and $f_k \in \mathcal{F}$ is a repair function. 
A conditional repair is compiled into generation procedure that returns a repair function; the repair function takes as input a possibly cleaned record $r$, along with its original uncleaned version $r_{orig}$:
{\small\begin{verbatim}
    def generate_repair(p, f):
      def repair(r, r_orig):
        if p(r): r = f(r)   
        return r 
      return apply
\end{verbatim}}

\begin{example}[Value Canonicalization]\sloppy
The following script canonicalizes different representations for Western United States:
{\small\begin{verbatim}
    def repair(r, r_orig):
      if r.region in ('USWest', 'USWESTERN'):
        r.region = 'USW'
      return r
\end{verbatim}}
\end{example}

\vspace{0.25em}
\begin{example}[Default Prediction]\sloppy
The following script represents a conditional prediction repair that predicts $false$ if the company name is missing.  Note that the predicate is applied on the original non-cleaned record. However the classifier takes as input the cleaned version.
{\small\begin{verbatim}
    def repair(r, r_orig):
      if r_orig.name == None:
        r.y = False
        return r
      return C(r)
\end{verbatim}}
\end{example}

\noindent Finally, let $L = (l_1,\cdots,l_n)$ be a sequence of conditional data and prediction repairs that \sys generates. 
$L$ is an element in a finite universe of possible repairs denoted by $\mathcal{L}=\mathcal{D}\times\mathcal{F}$.
To apply the repairs, \sys first partitions the $L$ into two subsequences $L^d = (l_i \in L | l_i$ is data repair$)$ and $L^p = (l_i \in L | l_i$ is prediction repair$)$.  During the training phase, we apply the data repairs in sequence over the training dataset prior to training the classifier:
\begin{align}
(X'_{train}, Y'_{train}) = \{L^d(r, r) | r \in (X_{train}, Y_{train}) \}\\
C = train((X'_{train}, Y'_{train})\\
L^d(r, r) = l_k(l_{k-1}(\cdots l_1(r, r), r) r) | l_i \in L^d
\end{align}

Finally, \sys constructs the final classifier $C_{L}$ by combining the prediction repairs $L^p$ with the trained classifier $C$.  It first identifies the last prediction repair $l^* \in L^p$ whose predicate matches the test record.  
$$l^* = \argmax_{l_i \in L^p \wedge l_i(r) = true} i$$
If no such prediction repair is found, \sys  returns the classifier prediction on the cleaned record, otherwise it applies $l^*$:
$$C_{L}(r) = \begin{cases}
    C(L^d(r, r))& \text{if } l^*\textrm{\ not\ found}\\
    l^*(L^d(r, r), r) & \text{otherwise}
\end{cases}$$

\subsection{Scope and Assumptions}
As a class of errors, we focus on domain integrity constraints, i.e., a set of allowed values in each attribute's domain--an error being defined as an attribute value not in this set.
Given a violation, we assume that each of the repair actions sets the attribute to an allowed value.
This assumption avoids a fixed-point iteration, also called the ``chase algorithm''~\cite{aho1979theory}, which repairs that cause additional errors.
This greatly simplifies the specification of $\mathcal{L}$ the set of possible data cleaning operations--in our experiments, $|\mathcal{L}|$ varied from $192$ to $1076$.
Next, we assume that each record in a relation corresponds to a single example (features and labels), and the analyst wants to learn a classifier that predicts labels from features.
Finally, we assume that the labels of the test data are clean since \sys relies on uncorrupted labels to estimate the model's accuracy.

\subsection{Problem Statement}
Given these assumptions, we define the repair selection problem:

\begin{problem}[\sys Repair Selection]\sloppy
Given $(X_{train}, Y_{train})$, $(X_{test}, Y_{test})$, a library of detector generators $\mathcal{D}$ and of repair functions $\mathcal{F}$, and a training procedure $train$, identify the optimal sequence $L^*$ of $B$ conditional repairs such that the resulting classifier $C_{L^*}$  maximizes prediction accuracy on $(X_{test}, Y_{test})$:
$$L^* = \argmax_{L \in \mathcal{D}\times\mathcal{F}} acc(C_L)$$
\end{problem}

Greedy solutions that select the top $B$ individual condition repairs will often fail since they might select highly correlated repairs (e.g., imputing a missing value with the mean, and the median).
Instead, it is desirable for an approach to take the mispredictions from previous conditional repairs into account.  This is the reason we applied a boosting-based approach towards selecting conditional repairs, described in the next section~\cite{schapire2003boosting}.

\begin{figure}\centering
\includegraphics[width=\columnwidth]{figures/workflow.png}
\caption{Offline (\orange{orange}) and online (\blue{blue}) workflows.}
\label{fig:workflow}
\end{figure}

Figure~\ref{fig:workflow} summarizes the training and prediction workflows given the optimal sequence of conditional repairs $\mathcal{L}^*$.  The \orange{orange line} depicts the training process, which first applies the conditional data repairs to the training dataset, and calls \texttt{train()} to generate classifier $C$.  The \blue{blue lines} depict how \sys generates a prediction for a test record: the classifier $C$ makes a prediction using the record cleaned by the conditional data repairs.  In addition, the conditional prediction repair checks the uncleaned test record to decide whether to return the classifier prediction or a default value.


\iffalse
\vspace{0.25em}
\noindent \textbf{Composition Rules: } \sys studies composing data cleaning operations to maximize predictive model accuracy trained on the cleaned data. Now, we need to formalize what it means to compose two data cleaning operations $l_1$ and $l_2$. Recall, that each data cleaning operation is a two-tuple of a predicate and cleaning action $(p, a)$. We define the composition of two operations $l_1=(p_1, a_1)$ and $l_2=(p_2, a_2)$ as generating two new operations (intersection of the predicates, and applying each respective action):
\ewu{is this correct?  should it be $p_1 \wedge p_2$, $p_i \wedge not p_2$
\[
l_1 \circ l_2 := \{(p_1 \wedge p_2, a_1), (p_1 \wedge p_2, a_2)\}
\]
Based on this definition, we can define a restricted library $\mathcal{L}_{\mid l_i}$ that is generated by $l_i$ composed with each of the members of the library:
\[\mathcal{L}_{\mid l_i} = \bigcup_{\forall j \ne i \in \mathcal{L}} l_i \circ l_j \]


\subsection{Extent of Supported Cleaning}
Simply put, \sys cannot support any data cleaning operation that changes the cardinality or labeling of the test dataset.
\reminder{TODO}.
\fi


% We provide an API for users to easily specify derived rules.

\iffalse
    The most basic type of predicate supported by \sys is a \emph{defined} predicate. 
    For example, we could enforce a rule on the running example dataset that every company has greater than 0 employees:
    \[ n\_emp < 0 \] 
    These rules can get more complex and span multiple attributes. For example, if we knew that there were no oil and natural gas companies in the northwest, we could enforce the following detection rule:
    \[ region == USNW \wedge industry == OIL-NG \]
    
    The error detector takes a set of these predicates $\{p_1,...,p_j\}$ and evaluates each one on the dataset to identify cells that are potentially dirty. There will be $j$ total sets of violations, and we denote each set of violations as $\{V_1,...,V_j\}$.
    
    
    
    In contrast to defined rules, \emph{derived rules} are rules that are learned from data.
    The module takes the loaded training dataset $F_{train}, P_{train}$ and returns a predicates $p$.
    An example of a derived rule is  statistical outlier detection, also called Quantitative Error Detection~\cite{hellerstein2008quantitative}.
    For example, one can scan $n\_emp$ calculate the mean value and standard deviation and derive the following rule:
    \[
    abs(n\_emp - mean) > 5*std
    \]
    We provide an API for users to easily specify derived rules.
\fi

\section{Repair Selection Algorithm}
The key insight of this paper is that the problem described in Section 3.3 can be addressed with statistical boosting.

\subsection{Overview of Boosting}
Ensemble methods construct predictions from combinations of predictors.
Boosting, a type of ensembling, is based on the observation that finding many ``weak learners'' is often significantly easier than finding a single, highly accurate predictor. 
The boosting algorithm calls this ``weak'' or ``base'' learning algorithm repeatedly feeding it a  weighting over the training examples.
Each time it is called, the base learning algorithm generates a new weak prediction rule, and after many rounds, the boosting algorithm must combine these weak learners into a single prediction rule that, hopefully, will be much more accurate than any one of the weak learners.

We will first introduce the classical AdaBoost algorithm for binary classifiers.
This is without a loss of generality since we can use an all-versus-one technique to handle multi-class classification.
The algorithm takes as input a training set of features and labels $(X,Y)$--assume that the labels are $\{-1, 1\}$.
AdaBoost calls a given weak learner repeatedly in a series of rounds. 
The algorithm re-weights the dataset after each round.
By training on a weighted dataset, we mean that it finds a learning from a set of permissible learners that maxmimizes the weighted accuracy.
For weighting function $W(x,y) \mapsto \mathbb{R}_+$:
\[
acc(C, W) = \frac{\sum_{x,y} W(x,y) \mathbf{1}(C((x, null)).y = y)}{\sum_{x,y} W(x,y) }
\]
Initially, all weights are set equally, but on each round, the weights of incorrectly classified examples are increased so that the learner is forced to focus on the hard examples in the training set.

Formally, the AdaBoost algorithm~\cite{freund1995desicion} proceeds as follows:
\begin{algorithm}
\KwData{(X, Y), $\alpha$}
Initialize $W^{(1)}_i = \frac{1}{N}$\\
\For{$t \in [1, T]$}{
  $C_t$ = Train weak learner on dataset weighed by $W^{t}_i$ \\
  $\epsilon_t$ = Calculate weighted classification error \\
  $\alpha_t = \ln(\frac{1-\epsilon_t}{\epsilon_t})$ \\
  $W^{(t+1)}_i \propto W^{(t)}_i e^{-\alpha_t y_i C_t(x_i)}$: down-weight correct predictions, up-weight incorrectly predictions.
}
\Return $C(x) = \text{sign}(\sum_t^T \alpha_t C_t(x) )$
\caption{AdaBoost Algorithm}
\label{alg:adaboost}
\end{algorithm}

\subsection{Why Boosting?}
The key difference from naive feature selection algorithms is that it selects over the space of models rather than the space of features.
If we have repair operations that cannot simply be represented as columnar operations (e.g., removing a record), this is a preferred solution.
Similarly, it makes few assumptions about how the attributes are aggregated into a model.

In our problem, each of the library elements will define a weak learner.
Given the dataset $R$, we can apply $l \in \mathcal{L}$ and then train the classifier to return $C_l$. 
The weak learners are evaluated on the clean test labels, which dictates weighting.
Modeling the selecting process as a statistical boosting allows us to make relatively few assumptions about the classifier and the data cleaning operations. 
Instead of having to reason about composing different data cleaning operations (and how compositions may affect accuracy), we are reasoning about a weighted consensus of classifiers trained with different data cleaning approaches.

\subsection{Repair Selection Algorithm}\label{s:boostalg}
The boosting algorithm weights the dataset depending on mispredictions, focusing future effort on the ensembles current mispredictions.
In each round, we find the $l \in \mathcal{L}$ that generates the classifier with highest test accuracy on the weighted data.
After selection, we recalculate the wights.
Repeat until $B$ cleaning operations are selected, by selecting the operation that performs best with updated weights.
The result is a new classifier $C_{clean}$ that is derived from the ensemble.
As before, without loss of generality we present the binary classification case with labels in $\{-1,1\}$.

\begin{algorithm}
\KwData{(X, Y), $\alpha$}
Initialize $W^{(1)}_i = \frac{1}{N}$\\

$\mathcal{L}$ generates a set of classifiers $\mathcal{C} \{C^{(0)}, C^{(1)},...,C^{(k)}\}$ where $C^{(0)}$ is the base classifier and $C^{(1)},...,C^{(k)}$ are derived from the cleaning operations.\\

\For{$t \in [1, T]$}{
  $C_t$ = Find $C_t \in \mathcal{C}$ that maximizes the weighted accuracy on the test set. 
  $\epsilon_t$ = Calculate weighted classification error on the test set
  $\alpha_t = \ln(\frac{1-\epsilon_t}{\epsilon_t})$ 
  $W^{(t+1)}_i \propto W^{(t)}_i e^{-\alpha_t y_i C_t(x_i)}$: down-weight correct predictions, up-weight incorrectly predictions.
}
\Return $C(x) = \text{sign}(\sum_t^T \alpha_t C_t(x) )$
\caption{Repair Selection Algorithm}
\label{alg:rsa}
\end{algorithm}

The algorithm has a few intuitive properties: (1) it prioritizes cleaning operations that improve performance, (2) if no such operations exist it does no worse than the base classifier, and (3) it is agnostic to the implementation of the classifiers.
The basic runtime of the algorithm is polynomial in both the number of cleaning operations and size of the dataset. In the next subsection, we will describe optimizations.

\begin{proposition}[Time Complexity]
The time complexity of Boost-and-Clean is $\mathbf{O}(k^2 N_{test} + k N_{train})$, where $k$ is the number of data cleaning operations, $N_{test}$ is the number of test tuples, and $N_{train}$ is the number of training tuples.
\end{proposition}

Boosting is well-understood statistically, and we can further bound the error on our clean test set (follows from~\cite{schapire2003boosting}):

\begin{proposition}[Error Bound]
For a budget of $B$ cleaning operations, the error rate of Boost-and-Clean on the test dataset decreases as $\mathbf{O}(e^{-2B})$.
\end{proposition}



\section{The Data Cleaning Library}
 In this section, we describe the architecture and API of the data cleaning library. 
 
\begin{figure}
\centering
\includegraphics[width=0.8\columnwidth]{figures/arch.pdf}
\caption{\sys system architecture.}
\label{f:arch}
\end{figure}

\subsection{Architecture}
The architecture of \sys includes three high-level components: (1) a loader, (2) error detector, and (3) repair action selector.
The loader takes in a weakly structured dataset (e.g., a CSV file) parses the file and returns a relation over a set of attributes and associated data types with each attribute.
Then, this relation is passed into an error detector.
The error detector identifies a candidate set of erroneous records and taxonomizes them into group of violated constraints.
Finally, associated with each of the violated constraint are repair actions (described in Section 3).
This modular architecture allows us to build a large number of library components, where users can specify additional rules for error detection without having to specify repair actions.
This section will overview the architecture of each of the components (including error detection) and the next section will go into the implementation details of the error detection module.

\subsection{Error Detection}
We noticed that many derived rules follow a similar structure.
First, they convert a row into a feature vector (in the above example, trivially projecting onto $num\_employees$).
Next, they apply a threshold (possibly multi-dimensional) on the feature vector.
We propose a modular architecture that has a single outlier detection technique (called Isolation Forests) at the end of the pipeline but allows for different types of featurization.
User defines a function that maps a row to a vector in $\mathbb{R}^d$ and the Isolation Forest identifies outliers in this space.
For a scalar feature in $\mathbb{R}$, the Isolation Forest degenerates into a threshold rule. 
We experimented with alternative outlier detection techniques (e.g., Minimum Covariance Determinant) but found that the Isolation Forest provided the best tradeoff between runtime and accuracy.

While the availability of such rules is a typical assumption in data cleaning~\cite{DBLP:conf/sigmod/ChuIKW16},  defining such rules requires extensive domain knowledge and is time-consuming.
One of our design objectives is to minimize the burden on data scientists, so we provide an initial library of rules which are described in the next section.
These rules are not meant to replace domain knowledge but rather to address routine problems seen across many different datasets.



\subsection{Loader}
The first step in using \sys is loading a dataset. \sys requires that that data is initially weakly structured, similar to the assumptions of SQLShare~\cite{howe2013sqlshare}. It assumes that each row corresponds to a record and attributes are delimited, but there are potentially missing values, the domain of possible attribute values are unknown, and the data types are unknown.
We implemented a schema-on-read loading module that takes as input a SQL table, CSV, or a text file.
This module returns a structured relation of tuples and inferred data types for each attribute (numerical, categorical, string, date, address).
We designed the type inference to be soft--allowing for errors to exist in the dataset.
The module automatically builds indices over the numerical and categorical attributes.
These indices will help optimize the error detection module.



\subsection{Cleaner}
Given a set of detected violations $\{V_1,...,V_j\}$, the cleaner assigns a tuple possible repair action to each $V_i$ (one if the data is in training, one if the data is in test).
Repairs are applied as a batch to each set of violations and not in a per-cell basis.
The available options are:
\begin{enumerate}
    \item \emph{Impute the mean (Train and Test): } Impute a cell in violation with the mean value of the attribute calculated over the training data excluding violated cells.
    \item \emph{Impute the mode (Train and Test): } Impute a cell in violation with the most frequent value of the attribute calculated over the training data excluding violated cells.
    \item \emph{Impute the median (Train and Test): }Impute a cell in violation with the median value of the attribute calculated over the training data excluding violated cells.
     \item \emph{Discard (Train Only): } Discard a row with a violated cell from the training dataset.
     \item \emph{Fail-Safe (Test Only): } Automatically predict the most-frequent label for a row with a violated cell.
\end{enumerate}

The cleaner's role is to learn an assignment of these actions to each of the violations.
The combination of the actions and the predicates define the data cleaning libary $\mathcal{L}$.








\section{Pre-populated Detectors}
We initialized the error detection module with defined and derived rules that are domain-agnostic.
We anticipate that users will build on these rules for specific use-cases.

\subsection{Observations from Real Data}
Abedjan et al. provide an extensive experimental evaluation of error detection techniques on real datasets~\cite{DBLP:journals/pvldb/AbedjanCDFIOPST16}.
The results of Abedjan et al. suggests that the datasets contain an interesting structure that can be exploited to pre-populate the library.
3 out of the 5 experimental datasets contained significant amounts of missing value constraints--which can largely be detected with sensible heuristics (e.g., hard-coded checks for NULL entries, No Alpha Numeric Characters, N/A, None).
On one of the remaining datasets, a Functional Dependency violation was correlated with erroneous numerical attribute values allowing it to be detected with quantitative methods.
And only the final dataset required a complex user-specified Denial Constraint to detect errors. 

In summary on 4 out of 5 datasets, a significant portion of errors, were essentially \emph{domain integrity} errors, i.e., a cell with a value not contained in the set of permissible values.
We argue that with appropriate featurization such constraints can be learned from data or can be identified by heuristics--and for the remaining errors user-specified constraints are necessary.
On 8 experimental datasets, our detector achieves a detection accuracy of 81\% of all of the errors found by hand-written rules.

\subsection{Heuristics}
In a first pass, we apply a collection of heuristics to detect obvious inconsistencies and type signature violations. These heuristics are implemented as \emph{defined} rules in the architecture.

\vspace{0.5em}
\noindent\textbf{Missing Values: }  We enumerate a set of patterns which commonly describe missing values in a database. This includes attributes that are actually \textsf{NULL}, empty strings, NaN, Inf, or otherwise lack alphanumeric characters.

\vspace{0.5em}
\noindent\textbf{Parsing/Type Errors: } Since each attribute is tagged with a type signature, we can evaluate if a value matches the type. For numerical values, this means that the entry can be parsed into a floating point number or an integer. For dates and addresses this means that the entry has a minimum of the required components (Month, Day, Year) or (Street, City, State).

\subsection{Detecting Quantitative Errors}
In addition to the heuristics, we project the dataset to retain just the numerical attributes and apply quantitative outlier detection techniques.
This module is implemented as a \emph{derived} rule in the architecture.
In numerical outlier detection the goal is to estimate the \emph{true} spread of a distribution and use that to threshold values that lie outside the spread.
The key challenge in outlier detection is that the outliers can affect one's estimate of a distribution's spread.

One approach is the Minimum Covariance Determinant (MCD), which is a robust estimator of the variance of a set of numbers, and has been used in a several recent works on numerical outlier detection (most notably MacroBase~\cite{bailis2016macrobase}).
We experimentally found that this approach is computation expensive and has a number of subtleties in implementation (like handling rank-deficient covariance matrices).
Instead, we found that a variant of Random Forest classification, called an Isolation Forest, was better suited for the problem.
The Isolation Forest isolates observations by randomly selecting a feature and then randomly selecting a split value between the maximum and minimum values of the selected feature.
Since recursive partitioning can be represented by a tree structure, the number of splittings required to isolate a sample is equivalent to the path length from the root node to the terminating node.
This path length, averaged over a forest of such random trees, is a measure of abnormality and our decision function.
Since these splits are axis aligned they can be efficiently compiled into threshold rules that can be evaluated on future data.

\subsection{Other Errors: Word2Vec Featurization}
However, quantitative errors are not a panacea and the difficulty in data quality research has been detecting errors that span multiple heterogenous attributes.
In principle, statistical outlier detection techniques can apply to featurized data.
The challenge is that naive featurization can explode the dimensionality of the feature space. Even a two-attribute relation with one numerical and one string-valued attribute, may have 1000s of features if one chooses a bag-of-words representation for the string-valued attribute.
The statistical power of outlier detection techniques rapidly diminish in the high-dimensional feature-spaces and the discrete distribution of feature vectors (e.g., bag-of-words) may violate the smoothness assumptions needed by the approaches.

One approach is to borrow recent results from Natural Language Processing using Neural Networks to first embed the records in a vector-space and then apply outlier detection techniques. The \textsf{word2vec} model \cite{mikolov2013distributed} is one such approach.
Using large amounts of unannotated plain text, \textsf{word2vec} learns relationships between words automatically with a Neural Network that predicts the occurrence of nearby words.
 Each word is assigned a vector in the vector space such that words that share common contexts (i.e., occur in the same document) in the corpus are located in close proximity to one another in the space.
 This vector space captures semantic relationships between words.
 
 We can adapt \textsf{word2vec} for featurizing records in a relation. Each record is treated as a document and each attribute is treated as word.
 The model is then trained using all of the records in the training dataset.
 Thus, for each attribute value we have a vector.
 To featurize a record, we concatenate these vectors together.
 Therefore, for each record $r$ there is an associated vector $r_v$.
  Like the NLP application, this vector space captures semantic relationships between records.
  We empirically find that applying the Isolation Forest outlier detection in this space leads to improved results.
We noticed that the quantitative outlier detection and word2vec both used the Isolation Forest as the main primitive, but with different featurizations.






%\section{The Data Cleaning Library}
 In this section, we describe the architecture and API of the data cleaning library. 
 
\begin{figure}
\centering
\includegraphics[width=0.8\columnwidth]{figures/arch.pdf}
\caption{\sys system architecture.}
\label{f:arch}
\end{figure}

\subsection{Architecture}
The architecture of \sys includes three high-level components: (1) a loader, (2) error detector, and (3) repair action selector.
The loader takes in a weakly structured dataset (e.g., a CSV file) parses the file and returns a relation over a set of attributes and associated data types with each attribute.
Then, this relation is passed into an error detector.
The error detector identifies a candidate set of erroneous records and taxonomizes them into group of violated constraints.
Finally, associated with each of the violated constraint are repair actions (described in Section 3).
This modular architecture allows us to build a large number of library components, where users can specify additional rules for error detection without having to specify repair actions.
This section will overview the architecture of each of the components (including error detection) and the next section will go into the implementation details of the error detection module.

\subsection{Error Detection}
We noticed that many derived rules follow a similar structure.
First, they convert a row into a feature vector (in the above example, trivially projecting onto $num\_employees$).
Next, they apply a threshold (possibly multi-dimensional) on the feature vector.
We propose a modular architecture that has a single outlier detection technique (called Isolation Forests) at the end of the pipeline but allows for different types of featurization.
User defines a function that maps a row to a vector in $\mathbb{R}^d$ and the Isolation Forest identifies outliers in this space.
For a scalar feature in $\mathbb{R}$, the Isolation Forest degenerates into a threshold rule. 
We experimented with alternative outlier detection techniques (e.g., Minimum Covariance Determinant) but found that the Isolation Forest provided the best tradeoff between runtime and accuracy.

While the availability of such rules is a typical assumption in data cleaning~\cite{DBLP:conf/sigmod/ChuIKW16},  defining such rules requires extensive domain knowledge and is time-consuming.
One of our design objectives is to minimize the burden on data scientists, so we provide an initial library of rules which are described in the next section.
These rules are not meant to replace domain knowledge but rather to address routine problems seen across many different datasets.



\subsection{Loader}
The first step in using \sys is loading a dataset. \sys requires that that data is initially weakly structured, similar to the assumptions of SQLShare~\cite{howe2013sqlshare}. It assumes that each row corresponds to a record and attributes are delimited, but there are potentially missing values, the domain of possible attribute values are unknown, and the data types are unknown.
We implemented a schema-on-read loading module that takes as input a SQL table, CSV, or a text file.
This module returns a structured relation of tuples and inferred data types for each attribute (numerical, categorical, string, date, address).
We designed the type inference to be soft--allowing for errors to exist in the dataset.
The module automatically builds indices over the numerical and categorical attributes.
These indices will help optimize the error detection module.



\subsection{Cleaner}
Given a set of detected violations $\{V_1,...,V_j\}$, the cleaner assigns a tuple possible repair action to each $V_i$ (one if the data is in training, one if the data is in test).
Repairs are applied as a batch to each set of violations and not in a per-cell basis.
The available options are:
\begin{enumerate}
    \item \emph{Impute the mean (Train and Test): } Impute a cell in violation with the mean value of the attribute calculated over the training data excluding violated cells.
    \item \emph{Impute the mode (Train and Test): } Impute a cell in violation with the most frequent value of the attribute calculated over the training data excluding violated cells.
    \item \emph{Impute the median (Train and Test): }Impute a cell in violation with the median value of the attribute calculated over the training data excluding violated cells.
     \item \emph{Discard (Train Only): } Discard a row with a violated cell from the training dataset.
     \item \emph{Fail-Safe (Test Only): } Automatically predict the most-frequent label for a row with a violated cell.
\end{enumerate}

The cleaner's role is to learn an assignment of these actions to each of the violations.
The combination of the actions and the predicates define the data cleaning libary $\mathcal{L}$.









%\input{cleaner.tex}
%\section{Synthesizer}
Given a the list of edits and cells, the synthesizer generates a Python program that given future unlabeled dirty data in the same schema will predict the the label. This Python program converts the proposed edits into a decision tree of if-else statements. It is impractical to retrain the neural network for each new test datum, and this cost is 3 orders of magnitude than SQL-based constraints. With code synthesis, initial experiments suggest that this optimization reduces evaluation time to within 1.7x of optimized SQL constraints. 

\section{Experiments}
In this section, we present the results of our experiments.
We first describe end-to-end use cases of \sys measuring accuracy and end-to-end runtime.
Then, we present a series of micro-benchmarks that evaluate each of the modules of \sys.

\subsection{Baselines and Methods}
To the best of our knowledge, there does not exist a comparable general purpose ML+Data Cleaning system to \sys in industry or academia.
We evaluate \sys against a number of baseline approaches inspired by solutions proposed in literature. 

\vspace{0.25em}\noindent\textbf{No Cleaning: } We train a model without any modification to the training or test data.

\vspace{0.25em}\noindent\textbf{Quantitative Only: } We train a model where only quantitative outliers in both training and test are imputed with a mean value. 

\vspace{0.25em}\noindent\textbf{Integrity Constraint: } We train a model where integrity constraints are corrected using a statistical distortion minimization metric as in~\cite{prokoshyna2015combining}.

\vspace{0.25em}\noindent\textbf{Quantitative + IC + Default Value: } We train a model where both quantitative and qualitative violations are corrected with a single action to impute the mean value (numerical) or mode value (categorical/string).

\vspace{0.25em}\noindent\textbf{Best Single: } We run \sys for $B=1$ and identify the best single data cleaning operation.

\vspace{0.25em}\noindent\textbf{Worst Single: } We run \sys for $B=1$ and identify the worst single data cleaning operation.

\vspace{0.25em}
In all of our experiments, we used standard classification models and featurization techniques from Python \textsf{sklearn}.
The classifiers were trained in Python 2.7 and timing experiments were run on an Amazon EC2 m4.16xlarge instance\footnote{64 virtual cpus and 256 GiB memory}.
To avoid overfitting, we carefully designed the accuracy evaluation experiments for \sys.
We use a ``doubly'' held out test dataset to address this problem, i.e., the test dataset that we use to optimize \sys is different from a completely unseen test dataset which is used to evaluate prediction accuracy.
We describe hyper-parameter settings for each technique in the text of each experiments.

To define the user-specified classifier, we user a Python \textsf{sklearn} Random Forest with problem specific depth and branch parameters.
The data is featurized using a pre-define library of standard featurizers (hot-one encoding for categorical data, bag-of-words for string data, numerical data as is), this is similar to the approach used in~\cite{DBLP:conf/sigmod/GokhaleDDNRSZ14}.

\subsection{End-to-End Accuracy}
In our first experiment, we evaluated the accuracy of \sys compared to the baselines.
We tried to minimize hyper-parameter tuning as much as possible to simulate a real-scenario with large dataset--where extensive tuning and parameter search might be expensive.
We looked at three classes of data: ML Competition Datasets, Data Analysis Scenarios, and Data From \company.

\subsection{ML Competition Datasets}
We downloaded 8 ML datasets used in Kaggle competitions and benchmarks in the UCI ML repository. 
These datasets are mostly clean as they have been extracted, structured, and published.
Nevertheless, they contain missing values, numerical outliers, and pattern errors (oddly formatted values).
The datasets are described below with brief descriptions and a description of the data errors:

\vspace{0.5em}

\reminder{TODO Write Description}

\subsection{Data Analytics}
We applied \sys to two raw datasets

\reminder{TODO Write Description}

\subsection{Company X Experiments}
We applied \sys to three datasets from Company X.

\reminder{TODO Write Description}


\begin{table*}[t]
\centering
\label{my-label}
\begin{tabular}{|l|r|r|r|r|r|r|r|r|r|}
\hline
ML Competition& NC & Q &	IC & Q+IC &	Best-1 &	Worst-1 &	BC-3 & BC5 & Rel. Improvement\\
\hline
USCensus	&0.85&	0.82&	0.86&	0.84&	0.87&	0.79&	0.88&	\blue{0.91} & +4.5\% \\
Emergency &	0.67&	0.72&	0.67&	0.72&	0.72&	0.66&	0.72&	\blue{0.75} & +4.7\%\\
Sensor	&0.92&	0.84&	0.93&	0.89&	0.92&	0.8&	\blue{0.94}&	0.94 & +1.3\%\\
NFL	&0.74&	0.74&	0.76&	0.75&	0.76&	0.74&	0.79&	\blue{0.82}& +5.1\%\\
EEG	&0.79&	0.82&	0.79&	0.83&	0.83&	0.7&	0.85&	\blue{0.89}& +6.8\%\\
Titanic	&0.83&	0.72&	0.83&	0.76&	0.83&	0.69&	0.83&	\blue{0.84}& +1.1\%\\
Housing	&0.73&	0.76&	0.73&	0.77&	0.77&	0.65&	\blue{0.81}&	0.76& +5.1\% \\
Retail	&0.88&	0.88&	0.91&	0.91&	0.91&	0.87&	0.94&	\blue{0.95}& +4.3\% \\
\hline
\hline
Data Analytics & NC & Q &	IC & Q+IC &	Best &	Worst &	BC-3 & BC5 & Rel. Improvement\\
\hline
FEC  & 0.62 & 0.53 & 0.61 & 0.57 & 0.71 & 0.51 & 0.74 & \blue{0.77} &  +8.4\% \\
Restaurant (Multiclass) & 0.42 & 0.42 & 0.58 & 0.68 & \blue{0.62} & 0.36 & 0.61 & 0.60 & (1.61)\% \\
\hline
\hline
Company X & NC & Q &	IC & Q+IC &	Best &	Worst &	BC-3 & BC5 & Rel. Improvement\\
\hline
Dataset 1 & & & & & & & & &\\
Dataset 2 & & & & & & & & &\\
Dataset 3 & & & & & & & & &\\
\hline
\end{tabular}
\caption{This table presents end-to-end accuracy results for each of the experimental datasets for \sys and alternatives. The results presented describe standard classification accuracy.}
\end{table*}

\subsection{Micro-Benchmarks}






\iffalse
\begin{figure}[t]
% \vspace{-5pt}
\centering
 \includegraphics[width=\columnwidth]{figures/exp1-bar.png}
 \caption{On 8 Machine Learning datasets, we evaluated the F1 score of different error detection techniques. (SQ) returns all values above 5 standard deviations from the mean. (SQ+MV) additionally applies heuristics to detect missing values. (SQ+MV+NN) additionally uses a Neural Network to find anomalous correlations between attributes. A single hyperparameter setting was used across all datasts. While SQ on its own performs poorly, SQ+MV+NN approaches the performance of the programatic rule-based approach--but without requiring a user to write the program.
 \label{fig:error}}
\end{figure}

\subsection{Initial Experiments}
To explore this question, we downloaded 8 benchmark datasets previously used in Machine Learning competitions. These datasets varied in size from 1460 records to 928,991 records.
We programatically cleaned these datasets up front through manual inspection and taxonomized the errors that we found (Figure \ref{fig:error}).
These datasets were already structured but had attribute errors, outliers, and inconsistent coding.

Confirming the results of Abedjan et al.~\cite{DBLP:journals/pvldb/AbedjanCDFIOPST16}, we found that a 
purely quantitative approach does not perform well in comparison to the rule-based approach on these datasets.
However, results are significantly improved when combined with heuristics that detect missing values. 
The performance gap is even further reduced when the detector additionally uses a Neural Network to learn how attributes correlate with each other, and detect anomolous correlations.

It is important to emphasize that these datasets represent a very specific domain, i.e., structured training datasets for ML.
The datasets are already in a structured schema and the only thing that an analyst has to worry about is handling inconsistent attribute values.
Presumably these datasets were also previously cleaned and extracted before they were publicly released.
Our initial experiment showed that for this class of datasets, reasonably accurate error detection is possible with minimal supervision and tuning.

\vspace{0.5em}
\noindent\textbf{Common Cleaning Patterns: }  If we can enumerate much of the errors in the dataset, the natural next question is whether we can automatically synthesize code to handle these errors. Since these datasets are publicly available, we surveyed ML code on github to understand how developers generally handled the issues in the particular datasets. The interesting insight is that data cleaning before ML model training has different design patterns that typical relational data cleaning. We found three common operations: (1) \emph{Feature Imputation. } impute the erroneous entry with a sensible default (consistent null symbol, mean value, most frequent value),  (2) \emph{Label Imputation. } when asked to predict a label for a dirty record return a null prediction or the most frequent label, and (3) \emph{Removal. } remove a dirty record from the training dataset. For each of the errors detected there are three options, and the the problem is select one of the three options in each case as to maximize the prediction accuracy.
\fi
\input{relatedwork.tex}
 \section{Conclusion and Future Work}
We have shown that it is possible, for data cleaning applications, to develop an automated cleaning system by casting the problem into a statistical boosting framework.  We have prototyped this idea in a new open-source data cleaning system called \sys\footnote{The system can be accessed at XXX (Anonymized for submission}.
We presented \sys, a new data cleaning system that detects errors in ML data and uses knowledge of the labels to adaptively select from a set of repair actions to maximize prediction accuracy.
We evaluated results on 8 ML datasets on Kaggle and the UCI repository with real data errors and compare to statistical anomaly detection techniques, constraint-based techniques, and the best single cleaner performance. In all 8 datasets, \sys had a higher test accuracy than alternatives. In addition, we have evaluated \sys on production datasets at a data science company and shown that \sys can automatically detect data errors and improve the prediction accuracy of the company's downstream model by up to $14\%$.  We also demonstrate how we can parallelize the inner-loop of the boosting operation, and on a 16-core machine \sys achieves a 9.7x speedup. Similarly, we show that building an inverted index can speed up operator selection time by 2.3x.

Our results are promising and they suggest several avenues for future research.
First, clearly the biggest limitation is the need for direct accuracy measure.
While this is available in many settings, we hope to relax this restriction in the future.
This might require a more complex ensembling technique than boosting.
Another interesting direction is considering infinite parametrized data cleaning libraries, and how selection from these libraries might work.
We also hope to relax the need for test labels by allowing other proxies, make it faster, evaluate on more datasets.
Of course, we also hope to continue industrial collaborations and real-world evaluations of our system.


% \input{impestimate.tex}

%\input{optimizer.tex}
%\section{Experiments}
In this section, we present the results of our experiments.
We first describe end-to-end use cases of \sys measuring accuracy and end-to-end runtime.
Then, we present a series of micro-benchmarks that evaluate each of the modules of \sys.

\subsection{Baselines and Methods}
To the best of our knowledge, there does not exist a comparable general purpose ML+Data Cleaning system to \sys in industry or academia.
We evaluate \sys against a number of baseline approaches inspired by solutions proposed in literature. 

\vspace{0.25em}\noindent\textbf{No Cleaning: } We train a model without any modification to the training or test data.

\vspace{0.25em}\noindent\textbf{Quantitative Only: } We train a model where only quantitative outliers in both training and test are imputed with a mean value. 

\vspace{0.25em}\noindent\textbf{Integrity Constraint: } We train a model where integrity constraints are corrected using a statistical distortion minimization metric as in~\cite{prokoshyna2015combining}.

\vspace{0.25em}\noindent\textbf{Quantitative + IC + Default Value: } We train a model where both quantitative and qualitative violations are corrected with a single action to impute the mean value (numerical) or mode value (categorical/string).

\vspace{0.25em}\noindent\textbf{Best Single: } We run \sys for $B=1$ and identify the best single data cleaning operation.

\vspace{0.25em}\noindent\textbf{Worst Single: } We run \sys for $B=1$ and identify the worst single data cleaning operation.

\vspace{0.25em}
In all of our experiments, we used standard classification models and featurization techniques from Python \textsf{sklearn}.
The classifiers were trained in Python 2.7 and timing experiments were run on an Amazon EC2 m4.16xlarge instance\footnote{64 virtual cpus and 256 GiB memory}.
To avoid overfitting, we carefully designed the accuracy evaluation experiments for \sys.
We use a ``doubly'' held out test dataset to address this problem, i.e., the test dataset that we use to optimize \sys is different from a completely unseen test dataset which is used to evaluate prediction accuracy.
We describe hyper-parameter settings for each technique in the text of each experiments.

To define the user-specified classifier, we user a Python \textsf{sklearn} Random Forest with problem specific depth and branch parameters.
The data is featurized using a pre-define library of standard featurizers (hot-one encoding for categorical data, bag-of-words for string data, numerical data as is), this is similar to the approach used in~\cite{DBLP:conf/sigmod/GokhaleDDNRSZ14}.

\subsection{End-to-End Accuracy}
In our first experiment, we evaluated the accuracy of \sys compared to the baselines.
We tried to minimize hyper-parameter tuning as much as possible to simulate a real-scenario with large dataset--where extensive tuning and parameter search might be expensive.
We looked at three classes of data: ML Competition Datasets, Data Analysis Scenarios, and Data From \company.

\subsection{ML Competition Datasets}
We downloaded 8 ML datasets used in Kaggle competitions and benchmarks in the UCI ML repository. 
These datasets are mostly clean as they have been extracted, structured, and published.
Nevertheless, they contain missing values, numerical outliers, and pattern errors (oddly formatted values).
The datasets are described below with brief descriptions and a description of the data errors:

\vspace{0.5em}

\reminder{TODO Write Description}

\subsection{Data Analytics}
We applied \sys to two raw datasets

\reminder{TODO Write Description}

\subsection{Company X Experiments}
We applied \sys to three datasets from Company X.

\reminder{TODO Write Description}


\begin{table*}[t]
\centering
\label{my-label}
\begin{tabular}{|l|r|r|r|r|r|r|r|r|r|}
\hline
ML Competition& NC & Q &	IC & Q+IC &	Best-1 &	Worst-1 &	BC-3 & BC5 & Rel. Improvement\\
\hline
USCensus	&0.85&	0.82&	0.86&	0.84&	0.87&	0.79&	0.88&	\blue{0.91} & +4.5\% \\
Emergency &	0.67&	0.72&	0.67&	0.72&	0.72&	0.66&	0.72&	\blue{0.75} & +4.7\%\\
Sensor	&0.92&	0.84&	0.93&	0.89&	0.92&	0.8&	\blue{0.94}&	0.94 & +1.3\%\\
NFL	&0.74&	0.74&	0.76&	0.75&	0.76&	0.74&	0.79&	\blue{0.82}& +5.1\%\\
EEG	&0.79&	0.82&	0.79&	0.83&	0.83&	0.7&	0.85&	\blue{0.89}& +6.8\%\\
Titanic	&0.83&	0.72&	0.83&	0.76&	0.83&	0.69&	0.83&	\blue{0.84}& +1.1\%\\
Housing	&0.73&	0.76&	0.73&	0.77&	0.77&	0.65&	\blue{0.81}&	0.76& +5.1\% \\
Retail	&0.88&	0.88&	0.91&	0.91&	0.91&	0.87&	0.94&	\blue{0.95}& +4.3\% \\
\hline
\hline
Data Analytics & NC & Q &	IC & Q+IC &	Best &	Worst &	BC-3 & BC5 & Rel. Improvement\\
\hline
FEC  & 0.62 & 0.53 & 0.61 & 0.57 & 0.71 & 0.51 & 0.74 & \blue{0.77} &  +8.4\% \\
Restaurant (Multiclass) & 0.42 & 0.42 & 0.58 & 0.68 & \blue{0.62} & 0.36 & 0.61 & 0.60 & (1.61)\% \\
\hline
\hline
Company X & NC & Q &	IC & Q+IC &	Best &	Worst &	BC-3 & BC5 & Rel. Improvement\\
\hline
Dataset 1 & & & & & & & & &\\
Dataset 2 & & & & & & & & &\\
Dataset 3 & & & & & & & & &\\
\hline
\end{tabular}
\caption{This table presents end-to-end accuracy results for each of the experimental datasets for \sys and alternatives. The results presented describe standard classification accuracy.}
\end{table*}

\subsection{Micro-Benchmarks}






\iffalse
\begin{figure}[t]
% \vspace{-5pt}
\centering
 \includegraphics[width=\columnwidth]{figures/exp1-bar.png}
 \caption{On 8 Machine Learning datasets, we evaluated the F1 score of different error detection techniques. (SQ) returns all values above 5 standard deviations from the mean. (SQ+MV) additionally applies heuristics to detect missing values. (SQ+MV+NN) additionally uses a Neural Network to find anomalous correlations between attributes. A single hyperparameter setting was used across all datasts. While SQ on its own performs poorly, SQ+MV+NN approaches the performance of the programatic rule-based approach--but without requiring a user to write the program.
 \label{fig:error}}
\end{figure}

\subsection{Initial Experiments}
To explore this question, we downloaded 8 benchmark datasets previously used in Machine Learning competitions. These datasets varied in size from 1460 records to 928,991 records.
We programatically cleaned these datasets up front through manual inspection and taxonomized the errors that we found (Figure \ref{fig:error}).
These datasets were already structured but had attribute errors, outliers, and inconsistent coding.

Confirming the results of Abedjan et al.~\cite{DBLP:journals/pvldb/AbedjanCDFIOPST16}, we found that a 
purely quantitative approach does not perform well in comparison to the rule-based approach on these datasets.
However, results are significantly improved when combined with heuristics that detect missing values. 
The performance gap is even further reduced when the detector additionally uses a Neural Network to learn how attributes correlate with each other, and detect anomolous correlations.

It is important to emphasize that these datasets represent a very specific domain, i.e., structured training datasets for ML.
The datasets are already in a structured schema and the only thing that an analyst has to worry about is handling inconsistent attribute values.
Presumably these datasets were also previously cleaned and extracted before they were publicly released.
Our initial experiment showed that for this class of datasets, reasonably accurate error detection is possible with minimal supervision and tuning.

\vspace{0.5em}
\noindent\textbf{Common Cleaning Patterns: }  If we can enumerate much of the errors in the dataset, the natural next question is whether we can automatically synthesize code to handle these errors. Since these datasets are publicly available, we surveyed ML code on github to understand how developers generally handled the issues in the particular datasets. The interesting insight is that data cleaning before ML model training has different design patterns that typical relational data cleaning. We found three common operations: (1) \emph{Feature Imputation. } impute the erroneous entry with a sensible default (consistent null symbol, mean value, most frequent value),  (2) \emph{Label Imputation. } when asked to predict a label for a dirty record return a null prediction or the most frequent label, and (3) \emph{Removal. } remove a dirty record from the training dataset. For each of the errors detected there are three options, and the the problem is select one of the three options in each case as to maximize the prediction accuracy.
\fi
%\input{relatedwork.tex}
%\input{discussion.tex}
%\section{Conclusion and Future Work}
We have shown that it is possible, for data cleaning applications, to develop an automated cleaning system by casting the problem into a statistical boosting framework.  We have prototyped this idea in a new open-source data cleaning system called \sys\footnote{The system can be accessed at XXX (Anonymized for submission}.
We presented \sys, a new data cleaning system that detects errors in ML data and uses knowledge of the labels to adaptively select from a set of repair actions to maximize prediction accuracy.
We evaluated results on 8 ML datasets on Kaggle and the UCI repository with real data errors and compare to statistical anomaly detection techniques, constraint-based techniques, and the best single cleaner performance. In all 8 datasets, \sys had a higher test accuracy than alternatives. In addition, we have evaluated \sys on production datasets at a data science company and shown that \sys can automatically detect data errors and improve the prediction accuracy of the company's downstream model by up to $14\%$.  We also demonstrate how we can parallelize the inner-loop of the boosting operation, and on a 16-core machine \sys achieves a 9.7x speedup. Similarly, we show that building an inverted index can speed up operator selection time by 2.3x.

Our results are promising and they suggest several avenues for future research.
First, clearly the biggest limitation is the need for direct accuracy measure.
While this is available in many settings, we hope to relax this restriction in the future.
This might require a more complex ensembling technique than boosting.
Another interesting direction is considering infinite parametrized data cleaning libraries, and how selection from these libraries might work.
We also hope to relax the need for test labels by allowing other proxies, make it faster, evaluate on more datasets.
Of course, we also hope to continue industrial collaborations and real-world evaluations of our system.

%\input{outlier.tex}
%\input{analysis.tex}
%\section{Experiments}
In this section, we present the results of our experiments.
We first describe end-to-end use cases of \sys measuring accuracy and end-to-end runtime.
Then, we present a series of micro-benchmarks that evaluate each of the modules of \sys.

\subsection{Baselines and Methods}
To the best of our knowledge, there does not exist a comparable general purpose ML+Data Cleaning system to \sys in industry or academia.
We evaluate \sys against a number of baseline approaches inspired by solutions proposed in literature. 

\vspace{0.25em}\noindent\textbf{No Cleaning: } We train a model without any modification to the training or test data.

\vspace{0.25em}\noindent\textbf{Quantitative Only: } We train a model where only quantitative outliers in both training and test are imputed with a mean value. 

\vspace{0.25em}\noindent\textbf{Integrity Constraint: } We train a model where integrity constraints are corrected using a statistical distortion minimization metric as in~\cite{prokoshyna2015combining}.

\vspace{0.25em}\noindent\textbf{Quantitative + IC + Default Value: } We train a model where both quantitative and qualitative violations are corrected with a single action to impute the mean value (numerical) or mode value (categorical/string).

\vspace{0.25em}\noindent\textbf{Best Single: } We run \sys for $B=1$ and identify the best single data cleaning operation.

\vspace{0.25em}\noindent\textbf{Worst Single: } We run \sys for $B=1$ and identify the worst single data cleaning operation.

\vspace{0.25em}
In all of our experiments, we used standard classification models and featurization techniques from Python \textsf{sklearn}.
The classifiers were trained in Python 2.7 and timing experiments were run on an Amazon EC2 m4.16xlarge instance\footnote{64 virtual cpus and 256 GiB memory}.
To avoid overfitting, we carefully designed the accuracy evaluation experiments for \sys.
We use a ``doubly'' held out test dataset to address this problem, i.e., the test dataset that we use to optimize \sys is different from a completely unseen test dataset which is used to evaluate prediction accuracy.
We describe hyper-parameter settings for each technique in the text of each experiments.

To define the user-specified classifier, we user a Python \textsf{sklearn} Random Forest with problem specific depth and branch parameters.
The data is featurized using a pre-define library of standard featurizers (hot-one encoding for categorical data, bag-of-words for string data, numerical data as is), this is similar to the approach used in~\cite{DBLP:conf/sigmod/GokhaleDDNRSZ14}.

\subsection{End-to-End Accuracy}
In our first experiment, we evaluated the accuracy of \sys compared to the baselines.
We tried to minimize hyper-parameter tuning as much as possible to simulate a real-scenario with large dataset--where extensive tuning and parameter search might be expensive.
We looked at three classes of data: ML Competition Datasets, Data Analysis Scenarios, and Data From \company.

\subsection{ML Competition Datasets}
We downloaded 8 ML datasets used in Kaggle competitions and benchmarks in the UCI ML repository. 
These datasets are mostly clean as they have been extracted, structured, and published.
Nevertheless, they contain missing values, numerical outliers, and pattern errors (oddly formatted values).
The datasets are described below with brief descriptions and a description of the data errors:

\vspace{0.5em}

\reminder{TODO Write Description}

\subsection{Data Analytics}
We applied \sys to two raw datasets

\reminder{TODO Write Description}

\subsection{Company X Experiments}
We applied \sys to three datasets from Company X.

\reminder{TODO Write Description}


\begin{table*}[t]
\centering
\label{my-label}
\begin{tabular}{|l|r|r|r|r|r|r|r|r|r|}
\hline
ML Competition& NC & Q &	IC & Q+IC &	Best-1 &	Worst-1 &	BC-3 & BC5 & Rel. Improvement\\
\hline
USCensus	&0.85&	0.82&	0.86&	0.84&	0.87&	0.79&	0.88&	\blue{0.91} & +4.5\% \\
Emergency &	0.67&	0.72&	0.67&	0.72&	0.72&	0.66&	0.72&	\blue{0.75} & +4.7\%\\
Sensor	&0.92&	0.84&	0.93&	0.89&	0.92&	0.8&	\blue{0.94}&	0.94 & +1.3\%\\
NFL	&0.74&	0.74&	0.76&	0.75&	0.76&	0.74&	0.79&	\blue{0.82}& +5.1\%\\
EEG	&0.79&	0.82&	0.79&	0.83&	0.83&	0.7&	0.85&	\blue{0.89}& +6.8\%\\
Titanic	&0.83&	0.72&	0.83&	0.76&	0.83&	0.69&	0.83&	\blue{0.84}& +1.1\%\\
Housing	&0.73&	0.76&	0.73&	0.77&	0.77&	0.65&	\blue{0.81}&	0.76& +5.1\% \\
Retail	&0.88&	0.88&	0.91&	0.91&	0.91&	0.87&	0.94&	\blue{0.95}& +4.3\% \\
\hline
\hline
Data Analytics & NC & Q &	IC & Q+IC &	Best &	Worst &	BC-3 & BC5 & Rel. Improvement\\
\hline
FEC  & 0.62 & 0.53 & 0.61 & 0.57 & 0.71 & 0.51 & 0.74 & \blue{0.77} &  +8.4\% \\
Restaurant (Multiclass) & 0.42 & 0.42 & 0.58 & 0.68 & \blue{0.62} & 0.36 & 0.61 & 0.60 & (1.61)\% \\
\hline
\hline
Company X & NC & Q &	IC & Q+IC &	Best &	Worst &	BC-3 & BC5 & Rel. Improvement\\
\hline
Dataset 1 & & & & & & & & &\\
Dataset 2 & & & & & & & & &\\
Dataset 3 & & & & & & & & &\\
\hline
\end{tabular}
\caption{This table presents end-to-end accuracy results for each of the experimental datasets for \sys and alternatives. The results presented describe standard classification accuracy.}
\end{table*}

\subsection{Micro-Benchmarks}






\iffalse
\begin{figure}[t]
% \vspace{-5pt}
\centering
 \includegraphics[width=\columnwidth]{figures/exp1-bar.png}
 \caption{On 8 Machine Learning datasets, we evaluated the F1 score of different error detection techniques. (SQ) returns all values above 5 standard deviations from the mean. (SQ+MV) additionally applies heuristics to detect missing values. (SQ+MV+NN) additionally uses a Neural Network to find anomalous correlations between attributes. A single hyperparameter setting was used across all datasts. While SQ on its own performs poorly, SQ+MV+NN approaches the performance of the programatic rule-based approach--but without requiring a user to write the program.
 \label{fig:error}}
\end{figure}

\subsection{Initial Experiments}
To explore this question, we downloaded 8 benchmark datasets previously used in Machine Learning competitions. These datasets varied in size from 1460 records to 928,991 records.
We programatically cleaned these datasets up front through manual inspection and taxonomized the errors that we found (Figure \ref{fig:error}).
These datasets were already structured but had attribute errors, outliers, and inconsistent coding.

Confirming the results of Abedjan et al.~\cite{DBLP:journals/pvldb/AbedjanCDFIOPST16}, we found that a 
purely quantitative approach does not perform well in comparison to the rule-based approach on these datasets.
However, results are significantly improved when combined with heuristics that detect missing values. 
The performance gap is even further reduced when the detector additionally uses a Neural Network to learn how attributes correlate with each other, and detect anomolous correlations.

It is important to emphasize that these datasets represent a very specific domain, i.e., structured training datasets for ML.
The datasets are already in a structured schema and the only thing that an analyst has to worry about is handling inconsistent attribute values.
Presumably these datasets were also previously cleaned and extracted before they were publicly released.
Our initial experiment showed that for this class of datasets, reasonably accurate error detection is possible with minimal supervision and tuning.

\vspace{0.5em}
\noindent\textbf{Common Cleaning Patterns: }  If we can enumerate much of the errors in the dataset, the natural next question is whether we can automatically synthesize code to handle these errors. Since these datasets are publicly available, we surveyed ML code on github to understand how developers generally handled the issues in the particular datasets. The interesting insight is that data cleaning before ML model training has different design patterns that typical relational data cleaning. We found three common operations: (1) \emph{Feature Imputation. } impute the erroneous entry with a sensible default (consistent null symbol, mean value, most frequent value),  (2) \emph{Label Imputation. } when asked to predict a label for a dirty record return a null prediction or the most frequent label, and (3) \emph{Removal. } remove a dirty record from the training dataset. For each of the errors detected there are three options, and the the problem is select one of the three options in each case as to maximize the prediction accuracy.
\fi
%\section{Conclusion and Future Work}
We have shown that it is possible, for data cleaning applications, to develop an automated cleaning system by casting the problem into a statistical boosting framework.  We have prototyped this idea in a new open-source data cleaning system called \sys\footnote{The system can be accessed at XXX (Anonymized for submission}.
We presented \sys, a new data cleaning system that detects errors in ML data and uses knowledge of the labels to adaptively select from a set of repair actions to maximize prediction accuracy.
We evaluated results on 8 ML datasets on Kaggle and the UCI repository with real data errors and compare to statistical anomaly detection techniques, constraint-based techniques, and the best single cleaner performance. In all 8 datasets, \sys had a higher test accuracy than alternatives. In addition, we have evaluated \sys on production datasets at a data science company and shown that \sys can automatically detect data errors and improve the prediction accuracy of the company's downstream model by up to $14\%$.  We also demonstrate how we can parallelize the inner-loop of the boosting operation, and on a 16-core machine \sys achieves a 9.7x speedup. Similarly, we show that building an inverted index can speed up operator selection time by 2.3x.

Our results are promising and they suggest several avenues for future research.
First, clearly the biggest limitation is the need for direct accuracy measure.
While this is available in many settings, we hope to relax this restriction in the future.
This might require a more complex ensembling technique than boosting.
Another interesting direction is considering infinite parametrized data cleaning libraries, and how selection from these libraries might work.
We also hope to relax the need for test labels by allowing other proxies, make it faster, evaluate on more datasets.
Of course, we also hope to continue industrial collaborations and real-world evaluations of our system.



%\bibliographystyle{abbrv}
%\scriptsize
%\fontsize{8.8pt}{9.9pt} \selectfont
\bibliographystyle{abbrv}
\bibliography{ref} 
\normalsize \selectfont
%\appendix
%\input{appendix.tex}

\end{document}
